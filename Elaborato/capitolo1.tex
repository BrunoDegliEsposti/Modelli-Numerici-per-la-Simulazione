\graphicspath{{./figures/capitolo1/}}
\lstset{inputpath = ./programs/capitolo1}
\pgfplotstableset{search path = ./tables/capitolo1}

\chapter{Introduzione}

Questo elaborato ha lo scopo di ripercorrere
gli argomenti più importanti del corso
\emph{Modelli numerici per la simulazione}, che ho frequentato
nell'anno accademico 2018/2019.
Secondo quanto visto a lezione, ho prestato 
particolare attenzione ai seguenti aspetti:
la formulazione e l'analisi di sistemi dinamici semplici ma significativi,
spesso tratti da modelli scientifici,
la loro discretizzazione tramite opportuni metodi numerici,
l'implementazione effettiva di questi metodi in ambiente MATLAB
e infine l'analisi critica dei dati numerici ottenuti dalle simulazioni
alla luce dei risultati teorici noti.
Come premessa a tutto ciò, mi è sembrato opportuno iniziare
l'elaborato con una riflessione sul concetto di errore nell'ambito
del calcolo scientifico.
%Questo elaborato ripercorre gli argomenti più importanti
%del corso \emph{Modelli numerici per la simulazione}, con particolare
%attenzione all'analisi di alcuni sistemi dinamici significativi,
%spesso tratti da modelli scientifici,
%a cui seguono una discretizzazione tramite opportuni metodi numerici,
%un'implementazione effettiva di questi metodi in ambiente MATLAB
%e un'analisi critica dei dati numerici ottenuti dalle simulazioni
%alla luce della teoria.
%Come premessa a tutto ciò, mi è sembrato opportuno iniziare
%l'elaborato con una riflessione sul concetto di errore nell'ambito
%del calcolo scientifico.

Prima, però, qualche precisazione di carattere tecnico.
Nonostante io abbia lavorato individualmente a questo elaborato,
ho preferito adottare la prima persona plurale in tutto il testo
per una questione di abitudine.
Sempre per la stessa ragione, ho preferito scrivere in inglese
i nomi delle funzioni e delle variabili nel codice MATLAB,
lasciando però i commenti in italiano.
La suddivisione dei capitoli segue fedelmente quella
del libro di testo di riferimento per il corso.

Il tipo di dato \code{table} ricorre spesso nel codice MATLAB,
perché l'ho usato per generare le tabelle e i grafici in modo automatico
all'interno del sorgente LaTeX (funzione offerta dal pacchetto \code{pgfplots}).
La colorazione automatica della sintassi è dovuta al
pacchetto \code{matlab-prettifier}, che estende il pacchetto \code{listings}.
Il codice che ho scritto è stato testato solo sulla release 2020b
di MATLAB, ma credo di non aver utilizzato alcuna funzionalità
particolarmente recente.

\section{Analisi dell'errore} \label{sec:errore}

L'affidabilità delle previsioni fornite dal calcolo scientifico
dipende in modo cruciale dalla capacità di comprendere,
stimare e mantenere sotto controllo gli errori durante l'intero processo di calcolo.
Senza pretese di completezza o generalità,
in questo capitolo andremo a elencare le principali fonti di errore.
Nessuna fonte è a priori trascurabile, perché
ciascuna di esse, nelle giuste circostanze, può risultare decisiva
nel proprio contributo all'errore complessivo.

A monte, è inevitabile che venga commesso un \emph{errore di modellizzazione}:
ogni teoria scientifica fa uso di modelli matematici, ma
un buon modello non è tanto un modello \emph{perfetto}%
\footnote{Infatti un tale modello, ammesso che esista,
non sarebbe meno complesso della realtà stessa.},
quanto un modello che, nella sua semplicità, è in grado di fornire descrizioni o
previsioni \emph{utili}, cioè affette da errori sufficientemente piccoli
per il contesto in cui il modello trova applicazione.
In altre parole, un buon modello è in grado di catturare
gli aspetti strettamente essenziali di un fenomeno
(in termini di grandezze, dinamiche, ecc),
andando invece a escludere tutto il resto.
%in modo consapevole quelli superflui

Una seconda fonte di errore è poi l'incertezza
a cui sono soggetti tutti i dati in ingresso al modello;
si parla in questo caso di \emph{errore sui dati}.
Non è possibile, infatti, misurare alcuna grandezza
naturale o sociale con precisione assoluta:
ogni quantità, sia essa determinata tramite inferenza statistica,
misurata sperimentalmente o fornita a sua volta da altri modelli,
è inevitabilmente soggetta a errore, e andrebbe pertanto rappresentata mediante una
distribuzione di probabilità, piuttosto che un singolo valore puntuale.
Questo approccio, ammesso che si sappia dire qualcosa su tali distribuzioni,
complica però di parecchio il modello; spesso ci si accontenta di tenere
traccia di intervalli di confidenza.
%(anche in senso generalizzato: ad esempio,
%l'incertezza di $\pm 0,05$mm associata alla misura con un calibro ventesimale, oppure
%la tolleranza del~5\% dichiarata da un costruttore di resistenze elettriche).
%Solitamente non spetta all'analista numerico occuparsi di questi tipi di errore.

Una terza fonte d'errore deriva dalle difficoltà computazionali
che si riscontrano durante l'applicazione del modello.
Quasi sempre, infatti, la soluzione del problema matematico
previsto dal modello non può essere calcolata in modo esatto
da un algoritmo che opera tramite un numero finito di operazioni elementari
sui dati iniziali; pertanto bisogna accontentarsi di
%un algoritmo che ne calcoli
un'approssimazione.%
\footnote{È un esempio celebre la soluzione per radicali
delle equazioni algebriche di grado 5 o superiore.}
In questo caso si parla di \emph{errore associato
al metodo numerico}. Questo è il tipo di errore che si manifesta,
ad esempio, ogni volta che un procedimento iterativo viene
arrestato prima di essere arrivato a convergenza, oppure
ogni volta che una serie numerica viene troncata, o magari quando
un integrale viene approssimato tramite una formula di quadratura.
Più nello specifico, in questo elaborato ci ritroveremo spesso ad analizzare
l'errore che si commette quando un ente matematico la cui natura è continua
%dalla natura continua
(un sistema dinamico, un oggetto geometrico, un operatore) viene sostituito
da un analogo ente discreto, sempre per esigenze computazionali: questo errore
è detto \emph{errore di discretizzazione}.

Un'ulteriore fonte di errore consiste nell'utilizzo di operazioni
aritmetiche inesatte all'interno di un algoritmo, al fine di velocizzarne
l'esecuzione e ridurne l'occupazione di memoria.
L'aritmetica esatta, laddove il suo uso sia possibile,
è infatti troppo onerosa da un punto di vista computazionale,
e per questo motivo viene utilizzata molto raramente nell'ambito del calcolo scientifico.
%(un uso possibile è la stima a posteriori dell'errore associato
%al metodo numerico).
Inoltre, sarebbe superfluo eseguire in modo esatto delle operazioni
aritmetiche su quantità comunque affette da errore,
secondo quanto detto nei punti precedenti. % (vedi punti precedenti).
Ad ogni modo, l'utilizzo dell'aritmetica inesatta non è privo di insidie:
bisogna assicurarsi che l'algoritmo scelto
non porti a un accumulo incontrollato dei piccoli%
\footnote{Anzi, a volte non proprio piccoli, come vedremo in seguito
parlando degli errori di cancellazione numerica.}
errori commessi da ogni operazione, detti \emph{errori dovuti all'aritmetica finita},
con conseguenze potenzialmente catastrofiche sull'accuratezza del risultato finale.
Lo studio di questo aspetto di un algoritmo è detto \emph{analisi di stabilità}:
formalmente un algoritmo di dice \emph{instabile} se, eseguito in aritmetica
inesatta, genera un errore sui dati in uscita di alcuni%
\footnote{Quanti, nella pratica? Non c'è una risposta univoca,
ma già quattro non ci sembrano trascurabili.}
ordini di grandezza maggiore dell'errore sui dati in ingresso al modello,
altrimenti si dice \emph{stabile}.
Non sempre è possibile trovare un'alternativa stabile a un algoritmo instabile:
talvolta è la natura stessa del problema matematico da risolvere che comporta
un'amplificazione massiccia degli errori sui dati in ingresso.
In questi casi, anche in assenza di errori associati al metodo numerico
o dovuti all'aritmetica finita, non si riuscirebbe comunque 
a determinare con precisione il valore in uscita dal modello;
si parla allora di \emph{problemi mal condizionati}.
Il fattore di proporzionalità tra gli errori sui dati in ingresso e gli
errori sui dati in uscita è detto \emph{numero di condizionamento del problema}.
Dunque, il numero di condizionamento di un problema pone un limite intrinseco
alla stabilità di qualunque algoritmo impiegato per risolvere il modello.
Un esempio estremo di problema mal condizionato è la simulazione
di un sistema dinamico caotico, come vedremo più avanti nel Capitolo 7.
Tornando all'aritmetica inesatta,
%
%questo tipo di errore è, in un certo senso, quello fondamentale, perché
%fornisce una limitazione inferiore a tutti gli altri tipi di errore.
%Inoltre stabilisce implicitamente se gli altri errori vadano calcolati
%in senso assoluto (nel caso dei numeri a virgola fissa) o relativo (nel caso
%dei numeri a virgola mobile).
%
tutte le simulazioni numeriche presenti in questo elaborato utilizzano
i numeri a virgola mobile con precisione doppia definiti dallo standard
IEEE 754 del 2008; le caratteristiche e le peculiarità di questo tipo
di aritmetica inesatta verranno descritte brevemente nel prossimo paragrafo.

A chiusura di questo paragrafo, invece, ricordiamo che esiste un'ultima
tipologia di errore, che è la più insidiosa e imprevedibile: l'\emph{errore umano}.
Nonostante i nostri sforzi, siamo certi che questo stesso elaborato non ne sia esente.

\section{Standard IEEE 754 per il calcolo in virgola mobile} \label{sec:ieee754}

%Per quanto possa essere interessante chiedersi quali numeri reali
%siano rappresentabili da un calcolatore (magari, idealizzato),

Capire quale sia il vero ostacolo che impedisce a un calcolatore (seppur idealizzato)
di rappresentare numeri reali arbitrari è senz'altro un problema affascinante.
%di rappresentare un numero reale qualunque è senz'altro un problema affascinante.
Tuttavia, nell'ambito dell'analisi numerica, ci sono ben altri motivi
(sostanzialmente, mancanza di tempo e memoria) per cui ci si accontenta,
nella pratica, di lavorare con un sottoinsieme finito dei numeri razionali,
distribuito in modo approssimativamente uniforme in scala logaritmica,
a cui ci riferiremo d'ora in poi come \emph{numeri di macchina}.
Nell'ambito del calcolo scientifico, i numeri di macchina più diffusi
sono i numeri a virgola mobile con precisione doppia definiti dallo standard IEEE 754.
Questi numeri, a parte qualche eccezione (lo zero, ad esempio),
sono della forma $\pm m 2^e$, con
\[
m \in \Q,
\quad m = \frac{n}{2^{52}},
\quad 2^{52} \leq n < 2^{53},
\quad e \in \Z,
\quad -1022 \leq e \leq 1023.
\]
La codifica di questi numeri di macchina richiede 64 bit, dei quali
il primo è dedicato al segno, i seguenti 11 all'esponente $e$
e gli ultimi 52 ai bit \emph{espliciti} della mantissa $m$.
La mantissa è composta anche da un bit \emph{implicito}, il più significativo,
che però è sempre 1 e pertanto non viene memorizzato.
La codifica dello zero (anzi, degli zeri, in base al segno),
degli infiniti, dei numeri denormalizzati e dei valori NaN (Not-a-Number)
aggiunge una certa complessità allo schema di codifica,
che però non è rilevante nel contesto di questo elaborato.

Piuttosto, è importante analizzare il massimo errore relativo che
si può commettere quando un numero reale $x$ viene
approssimato con il numero di macchina $r(x)$ più vicino
(in caso di ex aequo, viene scelto $n$ pari per convenzione).
Così definita, la \emph{funzione di arrotondamento} $r(x)$
soddisfa la stima
\begin{equation} \label{eq:stima}
\frac{\abs{x-r(x)}}{\abs{x}} \leq 2^{-53}
\quad \text{per ogni $x \in \pm [2^{-1022}, 2^{1023}]$}.
\end{equation}

Per quanto riguarda la definizione delle operazioni aritmetiche elementari
tra numeri di macchina (quelle, cioè, inesatte, d'ora in poi indicate
con l'aggiunta di un cerchietto), lo standard IEEE 754 richiede
che siano soddisfatte le seguenti identità (si parla di \emph{correct rounding}):
\begin{equation} \label{eq:identità}
a \oplus b = r(a+b),
\quad a \odot b = r(ab),
\quad a \ominus b = r(a-b),
\quad a \oslash b = r(a/b).
\end{equation}

La stima \eqref{eq:stima} e le identità \eqref{eq:identità} sono molto importanti,
perché insieme permettono di analizzare la stabilità di un algoritmo
mediante una tecnica \emph{perturbativa}. L'idea è quella di
equiparare l'aritmetica inesatta su numeri di macchina
all'aritmetica esatta su numeri reali affetti da un errore relativo
dell'ordine di $\varepsilon =  2^{-53} \approx 1.1 \cdot 10^{-16}$,
costante nota come \emph{epsilon di macchina}.
In alcuni contesti (ad esempio, in MATLAB), l'epsilon di macchina viene definito
in altro modo, cioè come la distanza che separa il numero 1 dal numero di
macchina successivo. In tal caso, $\varepsilon = 2^{-52}$,
ma l'ordine di grandezza rimane comunque $10^{-16}$.

A conclusione di questo paragrafo, passiamo in rassegna alcuni
problemi legati all'uso dei numeri di macchina di cui abbiamo
dovuto tenere di conto durante la scrittura dei programmi MATLAB
presenti nell'elaborato:
%problemi che si possono riscontrare nell'utilizzo di questi numeri di macchina:
\begin{itemize}
\item Le operazioni di somma e prodotto tra numeri di macchina sono commutative,
	ma non associative. Inoltre tra loro non vale la proprietà distributiva.
	Dunque la scelta di come associare e distribuire somme e prodotti
	in un algoritmo può avere delle conseguenze importanti
	durante la sua esecuzione in aritmetica inesatta; in alcuni casi
	fa addirittura la differenza tra un algoritmo stabile e uno instabile.
\item Quantità uguali in aritmetica esatta possono risultare diverse
	in aritmetica inesatta, e viceversa:
	\[
	r(1/3) \odot 3 \neq 1
	\qquad 1 \oplus 2^{-53} = 1
	\]
	Per questo motivo in un algoritmo ha solitamente poco senso
	controllare se due numeri di macchina siano uguali, perché l'esito
	può essere imprevedibile. Piuttosto, si preferisce
	rilassare la nozione di uguaglianza limitandosi a controllare se
	la distanza relativa tra i due numeri di macchina sia inferiore a
	un certo valore di soglia $s$, la cui scelta è però ovviamente
	delicata: bisogna trovare il giusto compromesso tra falsi positivi
	($s$ grande) e falsi negativi ($s$ piccola). Osserviamo inoltre che
	questa nozione rilassata di uguaglianza non è più transitiva.
\item L'operazione di somma $a+b$ tra numeri reali è mal condizionata
	per alcuni valori degli addendi $a$ e $b$. Si può dimostrare,
	infatti, che il numero di condizionamento relativo $k$ ottenuto
	fissando la norma infinito sul dominio $\R^2$ della funzione somma è
	\[
	\frac{2 \max\bigl\{|a|,|b|\bigr\}}{\abs{a+b}}.
	\]
	Purtroppo, non è raro che $\abs{a+b}$ sia di molti
	ordini di grandezza minore di $\max\bigl\{|a|,|b|\bigr\}$.
	Questo caso si verifica, per esempio,
	ogni volta che si approssima la derivata di una funzione
	in un punto mediante rapporto incrementale.
	Se $k$ è sufficientemente grande, allora la somma tra numeri di macchina
	è instabile e il fenomeno di amplificazione dell'errore relativo
	in questo caso prende il nome di \emph{cancellazione numerica}.
	Il prodotto tra numeri di macchina è invece sempre stabile,
	perché in questo caso $k \equiv 1$.
\item Dato che i numeri di macchina hanno 53 bit di mantissa
	(di cui uno implicito), tutti gli interi da $-2^{53}$ a $2^{53}$
	sono rappresentabili in modo esatto.
	Il numero $2^{53}+1$, invece, già non lo è più.
\end{itemize}

\section{Numeri di Stirling di seconda specie} \label{sec:stirling}

In numeri di Stirling sono quantità combinatorie importanti,
% spesso legati a problemi di enumerazione di partizioni o permutazioni.
legate all'enumerazione di particolari partizioni e permutazioni.
In questo paragrafo andremo a definire i numeri di Stirling di seconda specie
e a calcolarli in modo efficiente, mediante una formula ricorsiva.
Inoltre, cercheremo di mettere in luce la loro rilevanza nel contesto
del calcolo alle differenze finite, tema centrale di tutto l'elaborato
e in particolare del prossimo capitolo.
Innanzitutto, è utile la seguente definizione:
%Per farlo, occorre una definizione preliminare:

%	In un corso di combinatoria, i numeri di Stirling vengono solitamente
%	introdotti nel contesto della \emph{twelvefold way}, ossia un utile esercizio
%	di calcolo della cardinalità di insiemi di funzioni ideato da Giancarlo Rota,
%	padre della combinatoria moderna.
%	
%	Tuttavia, nell'ambito dell'analisi numerica, conviene interpretare questi numeri
%	in altro modo, ossia come coefficienti di cambio di base in $\Pi_k$,
%	spazio vettoriale dei polinomi reali di grado al più $k$.

\begin{defi}[Base delle pseudopotenze]
La base di $\R[x]$ formata dai polinomi
\begin{equation} \label{eq:pseudopotenze}
x^{(n)} \deq \prod_{i = 0}^{n-1} (x-i)
\quad \text{al variare di $n \in \N$}
\end{equation}
è detta \emph{base delle pseudopotenze} di $\R[x]$,
o anche \emph{base dei fattoriali decrescenti}.%
\footnote{In tutto l'elaborato, l'insieme $\N$ comprende anche lo zero.}
\end{defi}

Osserviamo che la base delle pseudopotenze, proprio come l'usuale
base dei monomi $x^n$, è composta da polinomi monici e di grado
strettamente crescente.
Sia $\Delta$ l'operatore di differenza in avanti unitaria, ossia
tale che $(\Delta f)(x) = f(x+1)-f(x)$.
Nel contesto del calcolo alle differenze finite,
l'operatore di differenza $\Delta$ svolge lo stesso
ruolo dell'operatore di derivata $\cramped{\frac{d}{dx}}$
nel contesto del calcolo differenziale.
Analogamente, la base delle pseudopotenze svolge lo stesso ruolo
della base dei monomi, come illustrato dal seguente teorema:

\begin{teor}
La matrice associata all'operatore lineare $\cramped{\frac{d}{dx}}$
una volta fissata in $\R[x]$ la base dei monomi è uguale alla matrice
associata a $\Delta$ una volta fissata in $\R[x]$ la base delle pseudopotenze.
In entrambi i casi, infatti, si ottiene la matrice $D$ tale che
$D_{ij} = i \:\! \delta_{i(j-1)}$.
\end{teor}

\begin{proof}
Segue dalla linearità di $\Delta$ e dall'identità $\Delta x^{(n)} = n x^{(n-1)}$.
\end{proof}

\begin{defi}[Numeri di Stirling]
I coefficienti $s_i^n$ della matrice
di cambio di base in $\R[x]$ dalla base
dei monomi alla base delle pseudopotenze sono detti
\emph{numeri di Stirling di prima specie (con segno)}.
I coefficienti $S_i^n$ della matrice associata al cambio di base inverso
sono detti \emph{numeri di Stirling di seconda specie}.%
\footnote{Osserviamo che, in via eccezionale,
gli indici di riga $n$ e di colonna $i$
delle matrici $s$ e $S$ partono da $0$ anziché $1$.}
Più formalmente, valgono per definizione le identità
\begin{equation} \label{eq:definizione-stirling}
x^{(n)} = \sum_{i=0}^{n} s_i^n x^i,
\quad
x^n = \sum_{i=0}^{n} S_i^n x^{(i)}
\quad
\text{per ogni $n \in \N$.}
\end{equation}
\end{defi}

La definizione appena data è ben posta, perché $\deg(x^n)=\deg(x^{(n)})$
e quindi non c'è bisogno di estendere le sommatorie \eqref{eq:definizione-stirling}
oltre l'indice $n$-esimo.
Le matrici $s$ e $S$ hanno dunque struttura triangolare inferiore,
perché l'unico valore che ha senso attribuire ai numeri di Stirling
quando l'indice $i$ supera $n$ è 0.

I numeri di Stirling di prima specie sono piuttosto semplici da calcolare:
basta espandere il prodotto nell'equazione \eqref{eq:pseudopotenze}.
In alternativa, cambiando la forma ma non la sostanza, si può partire
dall'identità $x^{(n)} = (x-n+1) x^{(n-1)}$ per dimostrare la formula ricorsiva
\begin{equation} \label{eq:ricorrenza-prima-specie}
s_i^n = s_{i-1}^{n-1} - (n-1)s_i^{n-1}.
\end{equation}

Per quanto riguarda invece il calcolo dei numeri di Stirling di seconda specie,
la definizione che abbiamo dato sembra suggerire la necessità di
invertire una sottomatrice finita di $s$,
il che avrebbe costo computazionale cubico.
In realtà, anche per i numeri di Stirling di seconda specie esiste
una formula ricorsiva analoga alla \eqref{eq:ricorrenza-prima-specie},
la quale permette di mantenere un costo quadratico.
\begin{teor}
I numeri di Stirling di seconda specie soddisfano la relazione di ricorrenza
\begin{equation} \label{eq:ricorrenza-seconda-specie}
S_i^n = S_{i-1}^{n-1} + i S_i^{n-1}
\end{equation}
con valori iniziali $S_0^0 = 1,
\; S_i^0 = 0 \text{ per ogni $i \geq 1$},
\; S_0^n = 0 \text{ per ogni $n \geq 1$.}$
\end{teor}

\begin{proof}
L'affermazione riguardante i valori iniziali segue dalla
struttura triangolare inferiore di $S$ e da considerazioni sui termini
noti dei polinomi nella seconda identità~\eqref{eq:definizione-stirling}.
Per quanto riguarda la relazione di ricorrenza,
vale la seguente catena di uguaglianze:
\begin{align*}
\sum_{i=0}^{n} S_i^n x^{(i)}
&= x^n
 = x x^{n-1}
 = x \sum_{i=0}^{n-1} S_i^{n-1} x^{(i)} \\
&= \sum_{i=0}^{n-1} S_i^{n-1} (x-i) x^{(i)}
	+ \sum_{i=0}^{n-1} S_i^{n-1} i x^{(i)}
 = \sum_{i=0}^{n-1} S_i^{n-1} x^{(i+1)}
	+ \sum_{i=0}^{n-1} i S_i^{n-1} x^{(i)} \\
&= \sum_{i=1}^{n} S_{i-1}^{n-1} x^{(i)}
	+ \sum_{i=0}^{n-1} i S_i^{n-1} x^{(i)}
 = \sum_{i=0}^{n} S_{i-1}^{n-1} x^{(i)}
	+ \sum_{i=0}^{n} i S_i^{n-1} x^{(i)}.
\end{align*}
Per concludere basta confrontare i coefficienti delle pseudopotenze
del primo e dell'ultimo termine della catena (stiamo implicitamente
usando l'indipendenza lineare delle pseudopotenze).
\end{proof}

Come preannunciato, grazie a questa formula ricorsiva è possibile
calcolare con $\Theta(n^2)$ operazioni elementari complessive%
\footnote{Notazione di Bachmann-Landau, con l'usuale significato
per i simboli $O$, $\Theta$ e $\Omega$.}
i numeri di Stirling di seconda specie fino a $S_n^n$.
Un possibile modo per farlo è illustrato nel Programma \ref{prog:stirling}.
A conclusione di questo paragrafo, commentiamo brevemente tale programma:

\begin{itemize}
\item Dato che gli indici in MATLAB partono da 1 anziché 0,
	per evitare confusione non andiamo a calcolare la riga
	e la colonna di $S$ corrispondenti a $n = 0$ e $i = 0$, rispettivamente.
	Scegliamo come nuovi valori iniziali $S_n^1 = 1$ per ogni $n \geq 1$.
\item In MATLAB, il tipo di dato numerico predefinito è quello a virgola
	mobile con precisione doppia IEEE 754, che come abbiamo visto
	non è in grado di gestire l'aritmetica tra interi in modo esatto
	al di fuori dell'intervallo $[-2^{53},2^{53}]$.
	La condizione d'uscita dal ciclo while serve pertanto a garantire
	che tutti i numeri di Stirling calcolati siano esatti.
	Quando ciò non è più possibile, l'algoritmo termina.
\item Quando allochiamo dello spazio in memoria per la matrice $S$
	con il comando \code{zeros} non c'è bisogno di superare le 55 righe.
	Infatti, applicando la \eqref{eq:ricorrenza-seconda-specie}
	alla seconda colonna di $S$, si ha
	\[
	S_2^n = S_1^{n-1} + 2S_2^{n-1} = 1 + 2S_2^{n-1}
	\]
	per cui si può dimostrare per induzione che $S_2^n = 2^{n-1}-1$.
	Se $n=55$, allora $2^{n-1} - 1 > 2^{53}$ e sappiamo con certezza
	che l'algoritmo non potrà procedere oltre.
\item Dato che $S$ è triangolare inferiore, possiamo evitare di calcolare
	tutti i valori al di sopra della diagonale. Inoltre, dato che riempiamo
	la matrice $S$ in ordine di riga crescente, evitiamo chiamate ricorsive
	su numeri di Stirling ancora da calcolare (anzi, evitiamo del tutto
	le chiamate ricorsive perché l'algoritmo è scritto in forma puramente
	iterativa). Questa idea, opportunamente generalizzata, è alla base
	della \emph{programmazione dinamica}.
\end{itemize}

%\afterpage{
\lstinputlisting[float=tp,
caption={Calcolo dei numeri di Stirling di seconda specie},
label=prog:stirling]{stirling.m}
%}








